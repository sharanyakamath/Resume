\documentclass{article}
\usepackage{scimisc-cv}
\usepackage{hyperref}
\hypersetup{
    colorlinks=true,
    linkcolor=blue,
    filecolor=magenta,      
    urlcolor=cyan,
}

\urlstyle{same}


%% These are custom commands defined in scimisc-cv.sty
\cvname{Sharanya Kamath}
\cvpersonalinfo{
+1 (413) 409-9509 \cvinfosep
98sharanya@gmail.com \cvinfosep
linkedin.com/in/sharanyakamath/ \cvinfosep
github.com/sharanyakamath
}

\begin{document}

% \maketitle %% This is LaTeX's default title constructed from \title,\author,\date

\makecvtitle %% This is a custom command constructing the CV title from \cvname, \cvpersonalinfo


% Applying To: Data Science - Interdisciplinary Group (MS in Data Science) Fall 2021 \hfill Application Id: 1000011473
\section{Education}

\cvsubsection{MS in CS with Data Science concentration}[Sept 2021 - Dec 2022]
[University of Massachusetts Amherst, USA][\textbf{GPA: 4.00 / 4.00 }]

\cvsubsection{BTech, Computer Science and Engineering}[July 2016 - June 2020]
[National Institute of Technology Karnataka, India] [\textbf{GPA: 8.77 / 10.00 }]


\section{Technical Skills}

\begin{itemize}
\item \textbf{Cloud:} AWS Cloud Practitioner certification, 3+ yrs designing, building, and maintaining AWS infrastructure for multiple projects. 
\item \textbf{Development:} Java, Python, Javascript, C++, Scala, C\#
\item \textbf{Frameworks:} Apache Spark, SpringBoot, Pandas, Tensorflow, Git
\item \textbf{Data Handling:} AWS Athena, DynamoDB, MongoDB, SQL, DataBricks, MATLAB, PowerBI
\item \textbf{Areas of Interest:} Cloud Architecture, Database Design, Big Data, Microservices Architecture.

\end{itemize}


\section{Work Experience}

\cvsubsection{JP Morgan Chase (New York City, NY, USA)}[Feb 2023 - Present]
[SDE 2, Commercial and Consumer Banking (CCB)][]

\begin{itemize}
\item Built AWS Simulator based on Moto library to make development and testing of AWS infrastructure possible locally.
\item Worked with Global Customer Platform to virtualize their services for consumer-side testing.
\item Responsible for building and maintaining a content management portal for test and release automation practices.
\item Engaged in North America Data Center migration effort by introducing smoke testing between migrating APIs.
\item Incorporated component testing using Hoverfly service virtualization as a best practice in CCB organization.
\end{itemize}

\cvsubsection{Amazon (Seattle, WA, USA)}[June 2022 - Aug 2022]
[SDE Intern, Supply Chain Optimization Technologies][]

\begin{itemize}
\item Built an end-to-end tool in Java Spring and ReactJS for querying the transaction path of an inventory item via AWS S3 buckets.
\item The Transaction S3 Query Tool helped in providing fast response to customers, reduced operational burdens for on-call employees, and raised root causes quickly in case of incidents.

\end{itemize}

\cvsubsection{Microsoft (Hyderabad, India)}[June 2020 - Aug 2021]
[Software Engineer, Cloud + AI Organization][]

\begin{itemize}
\item Responsible for ingestion, computation, and derivation of insights from big data streaming from various business sources.
\item Worked on a data modernization project which involved onboarding the operations of multiple teams onto Azure Data Lake.
\item Enabled the team to consolidate legacy systems and move towards futuristic BI solutions where contextual insights are available. 
\item Developed a Security Compliant Portal which ensured solid access provisioning flow for assets stored in Azure Data Lake.
\item Continuously worked on Azure cost and usage optimization, and strengthening of Services Data Lake platform capabilities.
\end{itemize}

\cvsubsection{Microsoft (Hyderabad, India)}[May 2019 - July 2019][Software Engineer Intern][]

\begin{itemize}
\item Proof of Concept on Azure Data Explorer: Analyzed new techniques of data ingestion, querying and storage. 
\item Achieved query time of 0.3sec for 27M rows, reduced ingestion time by 70\%, performed real-time visualization using PowerBI.
\item Built a recommendation system using Frequent Pattern Mining with Scala over Apache Spark framework.

\end{itemize}

% \cvsubsection{Here Technologies}[May 2018 - July 2018][R\&D Engineer Intern][]

% \begin{itemize}
% \item Optimization Planning through Metric Prediction for traffic sign identification. 
% \item Built series of 60\% to 99\% accurate regression models to predict time needed and tasks created during mapping of a region.

% \end{itemize}


\section{Selected Technical Projects}

\cvsubsection{Topic Segmentation in the Wild: Towards Segmentation of Semi-structured and Unstructured Chat Data}[Jan 2022 - May 2022]
[Graduate Student Research Project, Microsoft AI Development Acceleration Program (MAIDAP)][]
\begin{itemize}
\item Explored heirarchical BiLSTM and Cross-Segment BERT models for splitting chat data into meaningful contiguous segments.
% which correspond to a distinct topic or subtopic.
\item Analyzed model performance on two different types of chat data: semi-structured and unstructured. Concluded that performance is better on semi-structured data. Tried different loss functions: focal loss gave significantly better results.
\end{itemize}

\cvsubsection{Engagement Analysis of Students in Online Learning Environments }[Aug 2019 - April 2020]
[Undergraduate Major Project, Guide: Prof. Dr. Annappa B.][]
\begin{itemize}
\item Built an ensemble model of deep RNN and CNN models for expression recognition and gaze detection in Python.
\item Leveraged OpenFace 2.0 toolbox capabilities for face detection and feature extraction.
\end{itemize}

% \cvsubsection{Self-Tuned Proportional Integral Controller Enhanced (STPIE)}[Jan 2019 - April 2019]
% [Guide: Dr. Mohit P. Tahiliani, Course Project][]
% \begin{itemize}
% \item Designed a machine learning based self-tuning mechanism in NS-3 to reduce the queue delay in the network. 
% \item An improvement over the existing PIE algorithm for active queue management to solve the bufferbloat problem.
% \end{itemize}

% \cvsubsection{DNA Sequence Classification for Detection of Plasmid Fragments}[Feb 2019 - May 2019]
% [Guide: Dr. Shashidhar G Koolagudi, Course Project][]
% \begin{itemize}
% \item Built a Bidirectional LSTM based machine learning model to classify E.Coli bacterial DNA into chromosomal or plasmid.
% % \item  Performed character-level prediction to predict functionality of genes using only the sequence information (ATGTGT...). 
% \item Achieved a validation accuracy of 89\% and a cross-entropy loss of 0.29.
% \end{itemize}

\section{Publications}
\begin{itemize}
\item Ghosh, R., Kajal, H. S., Kamath, S., Shrivastava, D., Basu, S., and Srinivasan, S. Topic segmentation in the wild: Towards segmentation of semi-structured and unstructured chats. NeurIPS 2022 : ENLSP. \mathdoi{doi.org/10.48550/arxiv.2211.14954}
\item Kamath S, Singhal P, Jeevan G, Annappa B. Engagement Analysis of Students in Online Learning Environments. International Conference on Machine Learning and Big Data Analytics (ICMLBDA) 2021, Springer \mathdoi{doi.org/10.1007/978-3-030-82469-3\_4}

\end{itemize}

% \section{Key Courses }

% \begin{itemize}
% \item \textbf{Graduate level:} Machine Learning, Neural Networks, Systems for Data Science, Advanced Natural Language Processing, Reinforcement Learning, Algorithms for Data Science, Business Intelligence and Data Analytics, Internet Law and Policy
% \item \textbf{Undergraduate level:} DSA, Algorithmic Graph Theory, Digital Image Processing, Data Mining, Computer Vision, Computer Networks, Distributed Database Systems, Operating Systems, Compiler Design, Cloud Computing, DBMS, Software Engineering
% \item \textbf{Mathematics:} Statistics, Concrete Mathematics, Linear Algebra and Matrices, Discrete Mathematical Structures
% \end{itemize}


% \section{Extra-Curriculars}
% \begin{itemize}
% \item Computer Science Special Interests Group Head at ACM NITK, Algorithms Vice Chair at Web Enthusiasts Club NITK.
% \item 2\textsuperscript{nd} runner up from NITK in Microsoft Code.Fun.Do++ 2019 hackathon - Azure Blockchain E-Voting Challenge.
% \item Received AnitaB.Org scholarship to attend Grace Hopper Conference, 2022 in Orlando, FL.
% \end{itemize}


\end{document}
