\documentclass{article}
\usepackage{scimisc-cv}
\usepackage{hyperref}
\hypersetup{
    colorlinks=true,
    linkcolor=blue,
    filecolor=magenta,      
    urlcolor=cyan,
}

\urlstyle{same}


%% These are custom commands defined in scimisc-cv.sty
\cvname{Sharanya Shrinivas Kamath}
\cvpersonalinfo{
+91 9686191477 \cvinfosep
sherrykamath@gmail.com \cvinfosep
linkedin.com/in/sharanya-kamath-686277131/ \cvinfosep
github.com/sharanyakamath
}

\begin{document}

% \maketitle %% This is LaTeX's default title constructed from \title,\author,\date

\makecvtitle %% This is a custom command constructing the CV title from \cvname, \cvpersonalinfo


% Applying To: Data Science - Interdisciplinary Group (MS in Data Science) Fall 2021 \hfill Application Id: 1000011473
\section{Education}

\begin{itemize}
\item  B.Tech, Computer Science and Engineering, National Institute of Technology, Karnataka, India  \\
Graduated: June 2020 | CGPA: \textbf{8.77 / 10.00 }

\item Maharashtra State Board Class 12, Pace Junior Science College, Navi Mumbai, India \\
Graduated: March 2016 | Percentage: \textbf{93.8\%}

\item  CBSE Board Class 10, Apeejay School, Navi Mumbai, India \\
Graduated: March 2014 | Percentage: \textbf{95.6\%}
\end{itemize}


\section{Technical Skills}

\begin{itemize}

\item \textbf{Programming/DBMS/Dev:} C, C++, Python, Scala, Powershell, Java, MySQL, HTML, CSS, JavaScript, Django
\item \textbf{Frameworks:} Scikit-Learn, Pandas, Tensorflow, MLlib, Bootstrap, Git
\item \textbf{Data Science Tools:} Azure Data Explorer, Apache Spark, Azure Analysis Services, MATLAB, PowerBI, OpenCV, GNUPlot
\item \textbf{Area of Interest:} Machine Learning, Deep Learning, Data Science, Data Mining 

\end{itemize}

\section{Key Courses Undertaken}

\begin{itemize}
\item \textbf{Computer Science:} Data Structures and Algorithms, Design and Analysis of Algorithms, Algorithmic Graph Theory, Digital Image Processing, Data Warehousing and Data Mining, Computer Vision, Computer Networks, Distributed Database Systems, Operating Systems, Compiler Design, Cloud Computing, Database Management, Software Engineering, Theory Of Computation
\item \textbf{Mathematics:} Concrete Mathematics, Linear Algebra and Matrices, Discrete Mathematical Structures
\end{itemize}

\section{Work Experience}

\cvsubsection{Microsoft, India}[June 2020 - present]
[Software Engineer, Core Services Engineering and Operations][]

\begin{itemize}
\item Responsible for ingestion, computation and deriving insights from big data streaming from various business sources.
\item Involved in programming in Azure Databricks over Spark framework for parallel computation of entities to increase optimization. 
\item Working on a data modernization project which involves on-boarding the operations of multiple teams onto Azure Data Lake.
\end{itemize}

\cvsubsection{Microsoft, India}[May 2019 - July 2019][Software Engineer Intern][]

\begin{itemize}
\item Proof of Concept on Azure Data Explorer: Analyzed new techniques of data ingestion, querying and storage. 
\item Achieved query time of 0.3sec for 27M rows, reduced ingestion time by 70\%, performed real-time visualization using PowerBI.
\item Built a recommendation system using Frequent Pattern Mining with Scala over Apache Spark framework.

\end{itemize}

\cvsubsection{Here Technologies, India}[May 2018 - July 2018][R\&D Engineer Intern][]

\begin{itemize}
\item Optimization Planning through Metric Prediction for traffic sign identification. 
\item Built series of 60\% to 99\% accurate regression models to predict time needed and tasks created during mapping of a region.

\end{itemize}


\section{Selected Technical Projects}

\cvsubsection{Engagement Analysis of Students in Online Learning Environments}[Aug 2019 - April 2020]
[Guide: Prof. Dr. Annappa B., Undergraduate Major Project][]
\begin{itemize}
\item Built an ensemble model of deep RNN and CNN models for expression recognition and gaze detection in Python.
\item Leveraged OpenFace 2.0 toolbox capabilities for face detection and feature extraction.
\item Achieved an accuracy of 55.64\% on DAiSEE dataset, and a mean squared error of 0.0598 on In the Wild dataset.
\end{itemize}

\cvsubsection{Self-Tuned Proportional Integral Controller Enhanced (STPIE)}[Jan 2019 - April 2019]
[Guide: Dr. Mohit P. Tahiliani, Course Project][]
\begin{itemize}
\item Designed a machine learning based self-tuning mechanism in NS-3 to reduce the queue delay in the network. 
\item An improvement over the existing PIE algorithm for active queue management to solve the bufferbloat problem.
\end{itemize}

\cvsubsection{DNA Sequence Classification for Detection of Plasmid Fragments}[Feb 2019 - May 2019]
[Guide: Dr. Shashidhar G Koolagudi, Course Project][]
\begin{itemize}
\item Built a Bidirectional LSTM based machine learning model to classify E.Coli bacterial DNA into chromosomal or plasmid.
% \item  Performed character-level prediction to predict functionality of genes using only the sequence information (ATGTGT...). 
\item Achieved a validation accuracy of 89\% and a cross-entropy loss of 0.29.
\end{itemize}

\section{Extra-Curriculars}
\begin{itemize}
\item Computer Science Special Interests Group Head at ACM NITK, Algorithms Vice Chair at Web Enthusiasts Club NITK.
\item 2\textsuperscript{nd} runner up from NITK in Microsoft Code.Fun.Do++ 2019 hackathon - Azure Blockchain E-Voting Challenge.
\item Received AnitaB.Org scholarship to attend Grace Hopper Conference, India 2018.
\item Earned Visharad certificate (equivalent to a Bachelor's degree) in Bharatanatyam Indian classical dance in 2017.
\item Ranked 855 out of 1.3 million candidates in JEE Main Examination in 2016.
\end{itemize}


\end{document}
